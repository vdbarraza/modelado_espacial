% Options for packages loaded elsewhere
\PassOptionsToPackage{unicode}{hyperref}
\PassOptionsToPackage{hyphens}{url}
%
\documentclass[
  ignorenonframetext,
]{beamer}
\usepackage{pgfpages}
\setbeamertemplate{caption}[numbered]
\setbeamertemplate{caption label separator}{: }
\setbeamercolor{caption name}{fg=normal text.fg}
\beamertemplatenavigationsymbolsempty
% Prevent slide breaks in the middle of a paragraph
\widowpenalties 1 10000
\raggedbottom
\setbeamertemplate{part page}{
  \centering
  \begin{beamercolorbox}[sep=16pt,center]{part title}
    \usebeamerfont{part title}\insertpart\par
  \end{beamercolorbox}
}
\setbeamertemplate{section page}{
  \centering
  \begin{beamercolorbox}[sep=12pt,center]{part title}
    \usebeamerfont{section title}\insertsection\par
  \end{beamercolorbox}
}
\setbeamertemplate{subsection page}{
  \centering
  \begin{beamercolorbox}[sep=8pt,center]{part title}
    \usebeamerfont{subsection title}\insertsubsection\par
  \end{beamercolorbox}
}
\AtBeginPart{
  \frame{\partpage}
}
\AtBeginSection{
  \ifbibliography
  \else
    \frame{\sectionpage}
  \fi
}
\AtBeginSubsection{
  \frame{\subsectionpage}
}
\usepackage{amsmath,amssymb}
\usepackage{lmodern}
\usepackage{iftex}
\ifPDFTeX
  \usepackage[T1]{fontenc}
  \usepackage[utf8]{inputenc}
  \usepackage{textcomp} % provide euro and other symbols
\else % if luatex or xetex
  \usepackage{unicode-math}
  \defaultfontfeatures{Scale=MatchLowercase}
  \defaultfontfeatures[\rmfamily]{Ligatures=TeX,Scale=1}
\fi
\usetheme[]{AnnArbor}
\usecolortheme{dolphin}
\usefonttheme{structurebold}
% Use upquote if available, for straight quotes in verbatim environments
\IfFileExists{upquote.sty}{\usepackage{upquote}}{}
\IfFileExists{microtype.sty}{% use microtype if available
  \usepackage[]{microtype}
  \UseMicrotypeSet[protrusion]{basicmath} % disable protrusion for tt fonts
}{}
\makeatletter
\@ifundefined{KOMAClassName}{% if non-KOMA class
  \IfFileExists{parskip.sty}{%
    \usepackage{parskip}
  }{% else
    \setlength{\parindent}{0pt}
    \setlength{\parskip}{6pt plus 2pt minus 1pt}}
}{% if KOMA class
  \KOMAoptions{parskip=half}}
\makeatother
\usepackage{xcolor}
\newif\ifbibliography
\usepackage{graphicx}
\makeatletter
\def\maxwidth{\ifdim\Gin@nat@width>\linewidth\linewidth\else\Gin@nat@width\fi}
\def\maxheight{\ifdim\Gin@nat@height>\textheight\textheight\else\Gin@nat@height\fi}
\makeatother
% Scale images if necessary, so that they will not overflow the page
% margins by default, and it is still possible to overwrite the defaults
% using explicit options in \includegraphics[width, height, ...]{}
\setkeys{Gin}{width=\maxwidth,height=\maxheight,keepaspectratio}
% Set default figure placement to htbp
\makeatletter
\def\fps@figure{htbp}
\makeatother
\setlength{\emergencystretch}{3em} % prevent overfull lines
\providecommand{\tightlist}{%
  \setlength{\itemsep}{0pt}\setlength{\parskip}{0pt}}
\setcounter{secnumdepth}{-\maxdimen} % remove section numbering
\ifLuaTeX
  \usepackage{selnolig}  % disable illegal ligatures
\fi
\IfFileExists{bookmark.sty}{\usepackage{bookmark}}{\usepackage{hyperref}}
\IfFileExists{xurl.sty}{\usepackage{xurl}}{} % add URL line breaks if available
\urlstyle{same} % disable monospaced font for URLs
\hypersetup{
  pdftitle={Trabajo práctico:},
  pdfauthor={Verónica Barraza},
  hidelinks,
  pdfcreator={LaTeX via pandoc}}

\title{Trabajo práctico:}
\subtitle{Modelado Espacial}
\author{Verónica Barraza}
\date{}

\begin{document}
\frame{\titlepage}

\begin{frame}{Trabajo Práctico}
\protect\hypertarget{trabajo-pruxe1ctico}{}
\begin{block}{Modelado Espqcial}
\protect\hypertarget{modelado-espqcial}{}
\textbf{Objetivo}:

\begin{itemize}
\item
  Fundamentos de la teledetección.
\item
  Introducción al análsis de imagenes satelitáles con R
\item
  Introducción al análsis de series temporales satelitáles con R
\item
  Clasificación de imágenes satelitales con R
\end{itemize}
\end{block}
\end{frame}

\begin{frame}{Fundamentos de la teledetección}
\protect\hypertarget{fundamentos-de-la-teledetecciuxf3n}{}
\textbf{Teledetección} es la técnica que permite obtener información a
distancia de objetos sin que exista un contacto material. Para que ello
sea posible es necesario que, aunque sin contacto material, exista algún
tipo de interacción entre los objetos observados; situados sobre e la
superficie terrestre, marina o en la atmósfera; y un sensor situado en
una plataforma (satélite, avión, etc.).

En el caso la teledetección la interacción que se produce va a ser un
flujo de radiación que parte de los objetos y se dirige hacia el sensor.

\textbf{El sensoramiento remoto o teledetección es la práctica
científico-tecnológica que permite adquirir información de objetos o
sistemas sin estar en contacto físico directo con los mismos}
\end{frame}

\begin{frame}{Fundamentos de la teledetección}
\protect\hypertarget{fundamentos-de-la-teledetecciuxf3n-1}{}
Este flujo puede ser, en cuanto a su origen, de tres tipos:

\begin{itemize}
\item
  Radiación solar reflejada por los objetos(luz visible e infrarrojo
  reflejado)
\item
  Radiación terrestre emitida por los objetos (infrarrojo térmico)
\item
  Radiación emitida por el sensor y reflejada por los objetos (radar)
\end{itemize}

Las técnicas basadas en los dos primeros tipos se conocen como
teledetección pasiva y la última como teledetección activa.
\end{frame}

\begin{frame}{Ventajas y Desventajas}
\protect\hypertarget{ventajas-y-desventajas}{}
\begin{block}{Ventajas}
\protect\hypertarget{ventajas}{}
\begin{itemize}
\tightlist
\item
  Produce valores medios de una determinada área
\item
  Ofrece una amplia cobertura espacial y garantiza periodicidad
\item
  Permite el análisis temporal en la dimensión horizontal
\end{itemize}
\end{block}

\begin{block}{Desventajas}
\protect\hypertarget{desventajas}{}
\begin{itemize}
\tightlist
\item
  La mayoría de la E interactúa superficialmente con el objetivo
\item
  Dependiendo del sistema, la frecuencia de observación puede no ser
  adecuada
\item
  Radiación en el visible e IR no atraviesa las nubes
\item
  Las mediciones pueden ser imprecisas
\item
  Son necesarias mediciones in situ para validar los productos
\end{itemize}
\end{block}
\end{frame}

\begin{frame}{Espectro electromagnetico}
\protect\hypertarget{espectro-electromagnetico}{}
La radiación (solar reflejada, terrestre o emitida por el sensor y
reflejada) que llega de la superficie terrestre y que ha atravesado la
atmósfera, es almacenada en formato digital. Una vez recuperados los
datos en el centro de control del satélite, permitirán obtener
información acerca de la superficie terrestre y de la atmósfera. El tipo
de información que se obtiene dependerá de la longitud de onda en la que
el sensor capte radiación.

La REM de acuerdo a su longitud de onda, energía o frecuencia (se ha
dividido arbitrariamente en intervalos o bandas).

Cada material de la superficie terrestre (vegetación, rocas, minerales,
fauna, etc.), al estar compuesto por moléculas diversas y tener así
estructuras distintas, tendrá características electromagnéticas
diferentes, y pueden ser definidas, teóricamente, si podemos medir esas
diferencias (Soria y Matar, 2016). Los electrones ocupan órbitas o capas
discretas rodeando al núcleo, en una cantidad para cada átomo que está
determinada por la carga eléctrica del núcleo, que a su vez se debe a la
cantidad de protones que contenga ese núcleo. Cuando un electrón va de
una órbita externa a una órbita interior emite un fotón. La longitud de
onda de este fotón está determinada por las órbitas particulares entre
las que el electrón efectúa la transición. De esta forma, un espectro ,
que registra las longitudes de onda de los fotones, revela qué elementos
químicos conforman el objeto que se ha captado.

\begin{figure}
\centering
\includegraphics{https://static.uvq.edu.ar/mdm/teledeteccion/files/12.jpg}
\caption{Espectro electromagnetico}
\end{figure}
\end{frame}

\begin{frame}{Interacciones de la radiación electromagnética}
\protect\hypertarget{interacciones-de-la-radiaciuxf3n-electromagnuxe9tica}{}
Todos los objetos, con independencia de la radiación que emitan,
recibirán radiación emitida por otros cuerpos, principalmente del Sol.
En relación con el objeto sobre el que es emitida, esta radiación puede:

\begin{itemize}
\tightlist
\item
  reflejarse (la radiación es reenviada al espacio);
\item
  absorberse (la radiación incrementa la energía del objeto);
\item
  transmitirse (la radiación se transmite hacia abajo a otros objetos).
\end{itemize}

\begin{figure}
\centering
\includegraphics{https://static.uvq.edu.ar/mdm/teledeteccion/files/20.jpg}
\caption{Interacciones de la radiación electromagnética con la material}
\end{figure}
\end{frame}

\begin{frame}{Radiancia y Reflectancia}
\protect\hypertarget{radiancia-y-reflectancia}{}
Los datos que vienen almacenados en una imagen Landsat (o cualquier otra
imagen obtenida mediante un sensor óptico), son valores o niveles
digitales (ND). Dichos niveles digitales no representan de manera
directa ninguna variable biofísica y, por tanto, no es conveniente que
usted obtenga ningún índice espectral usando dichos valores ``crudos''.
La razón para no hacerlo es muy simple: los llamados ``índices
espectrales'' fueron desarrollados para trabajar con valores de
reflectancia espectral de la superficie terrestre. Los niveles digitales
no proporcionan dicha información. Por lo tanto, hay que convertir
dichos valores ND en valores de reflectancia.

Este proceso se realiza en dos etapas:

1- Conversión de ND a Radiancia (esta etapa se conoce como calibración
radiométrica)

2- Conversión de Radiancia a Reflectancia Aparente (es decir, el cálculo
de la reflectancia en el sensor)

Si, adicionalmente, se remueven los efectos atmosféricos, es posible
convertir la reflectancia en el sensor en reflectancia en la superficie.
En tal caso, se habrá realizado un proceso completo de corrección
atmosférica.
\end{frame}

\begin{frame}{Firma espectral o patrón espectral}
\protect\hypertarget{firma-espectral-o-patruxf3n-espectral}{}
Es la forma peculiar de reflejar o emitir energía de un determinado
objeto o cubierta.

Depende de las características físicas o químicas del objeto que
interaccionan con la energía electromagnética, y varía según las
longitudes de onda.

\begin{figure}
\centering
\includegraphics{https://images.squarespace-cdn.com/content/v1/521e95f4e4b01c5870ce81cf/1525455335128-7L6D5QJ4VJCH8V1ECQIA/FirmaEspectral.JPG}
\caption{Firma espectral}
\end{figure}
\end{frame}

\begin{frame}{Tipos de resoluciones}
\protect\hypertarget{tipos-de-resoluciones}{}
\begin{itemize}
\item
  ESPECTRAL: Se refiere al número de bandas y a la anchura espectral de
  las bandas
\item
  ESPACIA: Es la medida del objeto mas pequeño que puede ser distinguido
  sobre una imagen
\item
  TEMPORAL: Se refiere a cada cuanto tiempo recoge el sensor una imagen
  de un área en particular
\end{itemize}
\end{frame}

\begin{frame}{Indices de vegetación}
\protect\hypertarget{indices-de-vegetaciuxf3n}{}
\begin{itemize}
\item
  Los Índices de Vegetación son combinaciones de las bandas espectrales,
  cuya función es realzar la cubierta vegetal en función de su respuesta
  espectral y atenuar los detalles de otros componentes como el suelo,
  la iluminación, etc.
\item
  Cada Índice de Vegetación tiene sus limitaciones. El NDVI es sensible
  a los efectos del suelo y la atmósfera es, por ello, que se recomienda
  aplicar índices adicionales para un análisis más preciso de la
  vegetación.
\end{itemize}

\begin{block}{NDVI}
\protect\hypertarget{ndvi}{}
El más conocido es el Indice Normalizado de Vegetación (NDVI) :

\begin{itemize}
\item
  El NDVI es un índice cuantitativo de verdor que va desde -1 a 1, donde
  -1 representa un verdor mínimo o inexistente y 1 representa el verdor
  máximo.
\item
  Este índice se utiliza a menudo para una medida aproximada de salud,
  cobertura y fenología de la vegetación (etapa del ciclo de vida) en
  grandes extensiones territoriales.
\item
  También es habitual su uso en agricultura de precisión para estimar la
  salud de las plantaciones en determinadas parcelas
\item
  El cálculo del NDVI se realiza a partir del canal visible y el NIR
  reflejado por la vegetación, cuya ecuación es:
\end{itemize}

\[NDVI= (NIR- RED) /(NIR +RED)\]

\begin{block}{¿ Cómo interpretamos los resultados?}
\protect\hypertarget{cuxf3mo-interpretamos-los-resultados}{}
\begin{itemize}
\item
  La vegetación saludable (izquierda de la imagen) absorbe la mayor
  parte de la luz visible, y refleja una gran cantidad de energía de
  infrarrojo cercano.
\item
  La vegetación poco saludable o escasa (derecha de la imagen) refleja
  una luz más visible y una luz menos infrarroja
\item
  Los valores negativos corresponden a áreas con superficies de agua,
  estructuras artificiales, rocas, nubes, nieve; el suelo desnudo
  generalmente cae dentro del rango de 0.1 a 0.2; y las plantas siempre
  tendrán valores positivos entre 0.2 y 1.
\item
  El dosel de vegetación sano y denso debería estar por encima de 0.5, y
  la vegetación dispersa probablemente caerá dentro de 0.2 a 0.5. Sin
  embargo, es solo una regla general y siempre debe tener en cuenta la
  temporada, el tipo de planta y las peculiaridades regionales para
  saber exactamente qué significan los valores de NDVI.
\end{itemize}

\begin{figure}
\centering
\includegraphics{https://www.cursosteledeteccion.com/wp-content/uploads/2019/10/ndvi_que_es1.png}
\caption{NDVI}
\end{figure}
\end{block}
\end{block}

\begin{block}{EVI (Índice de Vegetación Mejorado)}
\protect\hypertarget{evi-uxedndice-de-vegetaciuxf3n-mejorado}{}
\begin{itemize}
\item
  El Enhanced vegetation index (EVI) o Índice de Vegetación Mejorado
  intenta expresar los efectos atmosféricos calculando la diferencia de
  radiancia entre las bandas del Azul y Rojo y nos permite monitorizar
  el estado de la vegetación en caso de altas densidades de biomasa.
\item
  El proceso resulta ser similar al cálculo del NDVI salvo que, en esta
  ocasión requerimos, además, la banda correspondiente al Azul del
  espectro visible. La fórmula de cálculo es la siguiente:
\end{itemize}
\end{block}
\end{frame}

\begin{frame}{Satelites ópticos}
\protect\hypertarget{satelites-uxf3pticos}{}
\begin{block}{LANDSAT}
\protect\hypertarget{landsat}{}
LANDSAT (LAND=tierra y SAT=satélite) fue el primer satélite enviado por
los Estados Unidos para el monitoreo de los recursos terrestres.
Inicialmente se le llamó ERTS-1 (Earth Resources Tecnology Satellite) y
posteriormente los restantes recibieron el nombre de LANDSAT.

\textbf{Características de las imágenes Landsat}

La misión Landsat ofrece diferentes imágenes satélite acotadas a
momentos temporales y bajo resoluciones de pixel diferente. Desde la
misión Landsat 1, hasta la actual Landsat 8, existe un repertorio
histórico de lo más variado. En función de la franja temporal de
análisis, dispones de las siguientes misiones Landsat:

\begin{itemize}
\tightlist
\item
  Landsat 1: Julio 1972-Enero 1978
\item
  Landsat 2: Enero 1975-Febrero 1982
\item
  Landsat 3: Marzo 1978-Marzo 1983
\item
  Landsat 4: Julio 1982-Diciembre 1993
\item
  Landsat 5: Enero 1984-Enero 2013
\item
  Landsat 7: Enero 1999-Actualidad
\item
  Landsat 8: Abril 2013-Actualidad
\item
  Landsat 9: septiembre de 2021- Actualidad
\end{itemize}

Las misiones Landsat 7 y Landsat 8 son las actualmente vigentes para la
descarga diaria de imágenes. La adquisición de sus imágenes, o
resolución temporal, es de 16 días. Por tanto, dispondrás de una nueva
imagen satélite actualizada para la misma zona de trabajo cada dos
semanas aproximadamente.

\textbf{Características de bandas}

Las imágenes de Landsat 8 están formadas por 10 bandas de trabajo + 1
banda pancromática cuyas resoluciones se encuentran en 15, 30 y 100
metros. Aunque su máxima resolución se encuentra en 30 metros, la banda
pancromática permite equiparar todas las bandas a una resolución de 15
metros a través de la técnica pansharpening, una particular forma de
remuestrear sus bandas a un tamaño homogéneo de 15 metros más pequeños.
En caso de no emplear esta técnica de refinado pancromático, el juego de
bandas te obligará a trabajar con resoluciones de 30 metros para todas
sus bandas y 100 metros para las bandas térmicas (TIR)

\begin{figure}
\centering
\includegraphics{http://www.gisandbeers.com/wp-content/uploads/2020/02/Bandas-Landsat-8.jpg}
\caption{Bandas}
\end{figure}
\end{block}
\end{frame}

\end{document}
