% Options for packages loaded elsewhere
\PassOptionsToPackage{unicode}{hyperref}
\PassOptionsToPackage{hyphens}{url}
%
\documentclass[
  ignorenonframetext,
]{beamer}
\usepackage{pgfpages}
\setbeamertemplate{caption}[numbered]
\setbeamertemplate{caption label separator}{: }
\setbeamercolor{caption name}{fg=normal text.fg}
\beamertemplatenavigationsymbolsempty
% Prevent slide breaks in the middle of a paragraph
\widowpenalties 1 10000
\raggedbottom
\setbeamertemplate{part page}{
  \centering
  \begin{beamercolorbox}[sep=16pt,center]{part title}
    \usebeamerfont{part title}\insertpart\par
  \end{beamercolorbox}
}
\setbeamertemplate{section page}{
  \centering
  \begin{beamercolorbox}[sep=12pt,center]{part title}
    \usebeamerfont{section title}\insertsection\par
  \end{beamercolorbox}
}
\setbeamertemplate{subsection page}{
  \centering
  \begin{beamercolorbox}[sep=8pt,center]{part title}
    \usebeamerfont{subsection title}\insertsubsection\par
  \end{beamercolorbox}
}
\AtBeginPart{
  \frame{\partpage}
}
\AtBeginSection{
  \ifbibliography
  \else
    \frame{\sectionpage}
  \fi
}
\AtBeginSubsection{
  \frame{\subsectionpage}
}
\usepackage{amsmath,amssymb}
\usepackage{lmodern}
\usepackage{iftex}
\ifPDFTeX
  \usepackage[T1]{fontenc}
  \usepackage[utf8]{inputenc}
  \usepackage{textcomp} % provide euro and other symbols
\else % if luatex or xetex
  \usepackage{unicode-math}
  \defaultfontfeatures{Scale=MatchLowercase}
  \defaultfontfeatures[\rmfamily]{Ligatures=TeX,Scale=1}
\fi
\usetheme[]{CambridgeUS}
\usecolortheme{seahorse}
% Use upquote if available, for straight quotes in verbatim environments
\IfFileExists{upquote.sty}{\usepackage{upquote}}{}
\IfFileExists{microtype.sty}{% use microtype if available
  \usepackage[]{microtype}
  \UseMicrotypeSet[protrusion]{basicmath} % disable protrusion for tt fonts
}{}
\makeatletter
\@ifundefined{KOMAClassName}{% if non-KOMA class
  \IfFileExists{parskip.sty}{%
    \usepackage{parskip}
  }{% else
    \setlength{\parindent}{0pt}
    \setlength{\parskip}{6pt plus 2pt minus 1pt}}
}{% if KOMA class
  \KOMAoptions{parskip=half}}
\makeatother
\usepackage{xcolor}
\newif\ifbibliography
\usepackage{graphicx}
\makeatletter
\def\maxwidth{\ifdim\Gin@nat@width>\linewidth\linewidth\else\Gin@nat@width\fi}
\def\maxheight{\ifdim\Gin@nat@height>\textheight\textheight\else\Gin@nat@height\fi}
\makeatother
% Scale images if necessary, so that they will not overflow the page
% margins by default, and it is still possible to overwrite the defaults
% using explicit options in \includegraphics[width, height, ...]{}
\setkeys{Gin}{width=\maxwidth,height=\maxheight,keepaspectratio}
% Set default figure placement to htbp
\makeatletter
\def\fps@figure{htbp}
\makeatother
\setlength{\emergencystretch}{3em} % prevent overfull lines
\providecommand{\tightlist}{%
  \setlength{\itemsep}{0pt}\setlength{\parskip}{0pt}}
\setcounter{secnumdepth}{-\maxdimen} % remove section numbering
\ifLuaTeX
  \usepackage{selnolig}  % disable illegal ligatures
\fi
\IfFileExists{bookmark.sty}{\usepackage{bookmark}}{\usepackage{hyperref}}
\IfFileExists{xurl.sty}{\usepackage{xurl}}{} % add URL line breaks if available
\urlstyle{same} % disable monospaced font for URLs
\hypersetup{
  pdftitle={Introducción a la teledetección},
  hidelinks,
  pdfcreator={LaTeX via pandoc}}

\title{Introducción a la teledetección}
\author{}
\date{\vspace{-2.5em}}

\begin{document}
\frame{\titlepage}

\begin{frame}{Fundamentos de la teledetección}
\protect\hypertarget{fundamentos-de-la-teledetecciuxf3n}{}
\textbf{Teledetección} es la técnica que permite obtener información a
distancia de objetos sin que exista un contacto material. Para que ello
sea posible es necesario que, aunque sin contacto material, exista algún
tipo de interacción entre los objetos observados; situados sobre e la
superficie terrestre, marina o en la atmósfera; y un sensor situado en
una plataforma (satélite, avión, etc.).

En el caso la teledetección la interacción que se produce va a ser un
flujo de radiación que parte de los objetos y se dirige hacia el sensor.

\textbf{El sensoramiento remoto o teledetección es la práctica
científico-tecnológica que permite adquirir información de objetos o
sistemas sin estar en contacto físico directo con los mismos}
\end{frame}

\begin{frame}{Fundamentos de la teledetección}
\protect\hypertarget{fundamentos-de-la-teledetecciuxf3n-1}{}
Este flujo puede ser, en cuanto a su origen, de tres tipos:

\begin{itemize}
\tightlist
\item
  Radiación solar reflejada por los objetos(luz visible e infrarrojo
  reflejado)
\item
  Radiación terrestre emitida por los objetos (infrarrojo térmico)
\item
  Radiación emitida por el sensor y reflejada por los objetos (radar)
\end{itemize}

Las técnicas basadas en los dos primeros tipos se conocen como
teledetección pasiva y la última como teledetección activa.
\end{frame}

\begin{frame}{Ventajas y Desventajas}
\protect\hypertarget{ventajas-y-desventajas}{}
\begin{block}{Ventajas}
\protect\hypertarget{ventajas}{}
\begin{itemize}
\tightlist
\item
  Produce valores medios de una determinada área
\item
  Ofrece una amplia cobertura espacial y garantiza periodicidad
\item
  Permite el análisis temporal en la dimensión horizontal
\end{itemize}
\end{block}

\begin{block}{Desventajas}
\protect\hypertarget{desventajas}{}
\begin{itemize}
\tightlist
\item
  La mayoría de la E interactúa superficialmente con el objetivo
\item
  Dependiendo del sistema, la frecuencia de observación puede no ser
  adecuada
\item
  Radiación en el visible e IR no atraviesa las nubes
\item
  Las mediciones pueden ser imprecisas
\item
  Son necesarias mediciones in situ para validar los productos
\end{itemize}
\end{block}
\end{frame}

\begin{frame}{Espectro electromagnetico}
\protect\hypertarget{espectro-electromagnetico}{}
La radiación (solar reflejada, terrestre o emitida por el sensor y
reflejada) que llega de la superficie terrestre y que ha atravesado la
atmósfera, es almacenada en formato digital. Una vez recuperados los
datos en el centro de control del satélite, permitirán obtener
información acerca de la superficie terrestre y de la atmósfera. El tipo
de información que se obtiene dependerá de la longitud de onda en la que
el sensor capte radiación.

La REM de acuerdo a su longitud de onda, energía o frecuencia (se ha
dividido arbitrariamente en intervalos o bandas)
\end{frame}

\begin{frame}{Espectro electromagnetico}
\protect\hypertarget{espectro-electromagnetico-1}{}
\begin{figure}
\centering
\includegraphics{https://concepto.de/wp-content/uploads/2019/05/espectro-electromagnetico-e1559140544472.jpg}
\caption{Espectro electromagnetico}
\end{figure}
\end{frame}

\begin{frame}{Firma espectral o patrón espectral}
\protect\hypertarget{firma-espectral-o-patruxf3n-espectral}{}
Es la forma peculiar de reflejar o emitir energía de un determinado
objeto o cubierta.

Depende de las características físicas o químicas del objeto que
interaccionan con la energía electromagnética, y varía según las
longitudes de onda.

\begin{figure}
\centering
\includegraphics{https://images.squarespace-cdn.com/content/v1/521e95f4e4b01c5870ce81cf/1525455335128-7L6D5QJ4VJCH8V1ECQIA/FirmaEspectral.JPG}
\caption{Firma espectral}
\end{figure}
\end{frame}

\begin{frame}{Tipos de resoluciones}
\protect\hypertarget{tipos-de-resoluciones}{}
\begin{itemize}
\item
  ESPECTRAL: Se refiere al número de bandas y a la anchura espectral de
  las bandas
\item
  ESPACIA: Es la medida del objeto mas pequeño que puede ser distinguido
  sobre una imagen
\item
  TEMPORAL: Se refiere a cada cuanto tiempo recoge el sensor una imagen
  de un área en particular
\end{itemize}
\end{frame}

\begin{frame}{Indices de vegetación}
\protect\hypertarget{indices-de-vegetaciuxf3n}{}
\begin{block}{NDVI}
\protect\hypertarget{ndvi}{}
\begin{figure}
\centering
\includegraphics{https://www.researchgate.net/publication/261712211/figure/fig2/AS:392467435278346@1470582936969/Figura-2-Curva-anual-del-indice-espectral-de-vegetacion-NDVI-o-EVI-y-sus-descriptores.png}
\caption{NDVI}
\end{figure}

El más conocido es el Indice Normalizado de Vegetación (NDVI) :

\begin{itemize}
\item
  El NDVI es un índice cuantitativo de verdor que va desde -1 a 1, donde
  -1 representa un verdor mínimo o inexistente y 1 representa el verdor
  máximo.
\item
  Este índice se utiliza a menudo para una medida aproximada de salud,
  cobertura y fenología de la vegetación (etapa del ciclo de vida) en
  grandes extensiones territoriales.
\item
  También es habitual su uso en agricultura de precisión para estimar la
  salud de las plantaciones en determinadas parcelas
\item
  El cálculo del NDVI se realiza a partir del canal visible y el NIR
  reflejado por la vegetación, cuya ecuación es:
\end{itemize}

\[ NDVI= (NIR- RED) /(NIR +RED)\]

\begin{itemize}
\item
  La vegetación saludable (izquierda de la imagen) absorbe la mayor
  parte de la luz visible, y refleja una gran cantidad de energía de
  infrarrojo cercano.
\item
  La vegetación poco saludable o escasa (derecha de la imagen) refleja
  una luz más visible y una luz menos infrarroja
\end{itemize}
\end{block}
\end{frame}

\end{document}
